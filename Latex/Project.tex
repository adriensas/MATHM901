\documentclass[a4paper,12pt]{article}
	\usepackage{amsmath}
	\usepackage{amsthm}
	\usepackage{amssymb}

\theoremstyle{definition}
\title{Kerr and Schwarzschild spacetimes derived differently}
\author{Adrien Sasportes}

\begin{document}

\maketitle
\pagenumbering{arabic}
\newtheorem{definition}{Definition}

\section{Introduction}

\section{Geometry and space-time}
This project focus on general relativity and require some knowledge on the subject.
General Relativity is an adequation between gravity and the curvature of the space-time.
Thus it seems normal to start it with a short introduction to geometry and space-time.
Before talking about curvature of the space-time we must first start to define it.
\subsection{Manifold and metric}
In order to define space time, we first need to define a manifold.
A manifold is a topological space that locally resembles Euclidian space near each point.
More formaly :
\begin{definition}
	\textbf{Manifold} :An $n$-dimensional smooth manifold $\mathcal{M}$ is a set together with a collection of coordinates $\{ X^i_{(\alpha)}\}$
	with $\alpha \in \mathbb{N}$ and $i=1,2,...,n$. These satisfy :
	\begin{enumerate}
		\item Every point $p\in\mathcal{M}$ is also in at least one $\{\Omega_\alpha\}$.
		This mean that the collection $\{\Omega_\alpha\}$ will contain every point $p$,
		we say that the manifold is covered by $\{\Omega_\alpha\}$.
		\item For each $\alpha$ the coordinates $\{ X^i_{(\alpha)}\}$ are bijective functions $X^i_{(\alpha)}:\Omega_\alpha \mapsto O_\alpha \subset \mathbb{R}^n$
		where the $O_\alpha$ are also open.
		\item If any $U_\alpha$ and $U_\beta$ overlap, then the coordinates $\{ X^i_{(\alpha)}\}$ and $\{ X^i_{(\beta)}\}$
		are related by a coordinarte transformation on the intersection $U_\alpha \cap U_\beta$.
	\end{enumerate}
\end{definition}

\section{Vaidya type metric in GR}

\section{Deriving Schwarzschild differently}
In this section we will be interested in trying to find Schwarzschild differently using the general Kerr-Shild form :
\begin{equation*}
	g_{ij}=\eta_{ij}+2H(r)k_ik_j
\end{equation*}
The Schwarzschild solution is a really important solutionin general relativity. It was discovered a little more than a month after Einstein had formulated General Relativity and it describes the gravitationnal field outside a static spherical mass.
This solution can be used as an approximation for slowly rotating spherical astronomical object, like the Earth.
\subsection{Classical way of finding the Schwarzschild solution}
The usual way of finding the Schwarzschild metric usually start by writing the most general symmetric metric in vacuum :
\begin{equation*}
	ds^2=-e^{A(r)}dt^2+e^{B(r)}dr^2+r^2d\Omega^2
\end{equation*}
Then we compute the Einstein tensor, and we solve the Einstein vacuum field equation $G_{ab}=0$ by taking $G_{00}=0$ and $G_{11}=0$. Then we obtain the following equations :
\begin{align*}
	e^B+rB'-1&=0\\
	-e^B+rA'+1&=0
\end{align*}
From which we get :
\begin{equation*}
	e^A=e^{-B}=1-\frac{C}{r}
\end{equation*}
And thus we have the Schwarzschild solution :
\begin{equation*}
	ds^2=-(1-\frac{C}{r})dt^2+(1-\frac{C}{r})^{-1}dr^2+r^2d\Omega^2
\end{equation*}
This method of finding the Schwarzschild solution is work perfectly way but still it is always interesting to find an alternative way to find it.
\subsection{Elements of the Kerr-Shild for the Schwarzschild case}
We now want to focus on a metric of the following form $g_{ij}=\eta_{ij}+2H(r)k_ik_j$.
First in the Schwarzschild case we consider a spherical object so we consider some $r$ such that $r^2=x^2+y^2+z^2$.
$\eta_{ij}$ is known as the Minkowski metric defines as $\eta_{ij}=diag(+1,+1,+1,-1)$
As for the $H$ we guess that it would only depend on $r$.
We define $k_i=(-1,\frac{x}{r},\frac{y}{r},\frac{z}{r})$, a null geodesic.
We notice that $k_ik^i=0 <=> r^2=x^2+y^2+z^2$.
We notice that the spatial part of $k_i$ is a radial components but is the metric symmetric?
\subsection{Einstein vacuum field equation}
From the formula we derived for $G_{ij}$ for metric of this form we can fairly easily obtain the Einstein tensor.
Looking at $G_{i0}, i=1,2,3$ we get :
\begin{equation}
	G_{i0}=-\frac{8 H(r) (r^2 H'(r)+rH(r))}{(r^2)^{3/2}}
\end{equation}
Our goal is to find $H(r)$ that solves the Einstein vacuum field equation $G_{ij}=0$ which in our case is $G_{i0}=0, i=1,2,3$ or :
\begin{equation}
	-\frac{8 H(r) (r^2 H'(r)+rH(r))}{(r^2)^{3/2}}=0
\end{equation}
Which in general is equivalent to :
\begin{equation}
	rH'(r)+H(r)=0
\end{equation}
This equation is quite nice and can easily be solved as follow :
\begin{align*}
	rH'(r)+H(r)&=0\\
	r\frac{d}{dr}H(r)&=-H(r)\\
	\frac{1}{H(r)}\frac{d}{dr}H(r)&=-\frac{1}{r}\\
	\frac{1}{H(r)}dH(r)&=-\frac{1}{r}dr \;\\
	&\text{, integrating,}\\
	log(H(r))&=-\log{r}+C \;\\
	&\text{, with C constant}\\
	H(r)&=\frac{C}{r}
\end{align*}
Giving us :
\begin{equation}
	g_{ij}=\eta_{ij}+\frac{2C}{r}k_ik_j
\end{equation}
This is the Schwarzschild solution. As it is not so obvious in this form, in the next sub-section, we will put in back in the "classical form".
\subsection{Back to the classical form of the Schwarzschild solution}
In order to be able to find the classical Schwarzschild form we should first write the line element of the metric we found explcitly :
\begin{equation*}
	ds^2=g_{ij}dX^idX^j
\end{equation*}
\begin{equation}
	ds^2=-dt^2+dx^2+dy^2+dz^2+\frac{2C}{r}(-dt+\frac{x}{r}dx+\frac{y}{r}dy+\frac{z}{r}dz)^2
\end{equation}
In order to find the classical form of the Schwarzschild metric, we will this line element in spherical coordinates $X=\{t,r,\theta,\phi\}$ define as :
\begin{align*}
	&t=t\\
	&r^2=x^2+y^2+z^2
\end{align*}
\begin{align}
	&x=r\sin{\theta}\cos{\phi}\label{coord-transform-1}\\
	&y=r\sin{\theta}\sin{\phi}\\
	&z=r\cos{\theta}\label{coord-transform-3}
\end{align}
As we have constructed $g_{ij}$ to be the sum of the Minkowski metric and another element we can immediatly replace the Minkowski part of our metric with the Minkowski metric in spherical coordinates :
\begin{equation} \label{line-elem-schwarz}
	ds^2=-dt^2+dr^2+r^2d\Omega^2\;+\frac{2C}{r}(-dt+\frac{x}{r}dx+\frac{y}{r}dy+\frac{z}{r}dz)^2
\end{equation} 
Now let's focus on $\frac{2C}{r}(-dt+\frac{x}{r}dx+\frac{y}{r}dy+\frac{z}{r}dz)^2$.\\
First we want to write $dx$, $dy$, $dz$ as a linear sum of $dr$, $d\theta$, $d\phi$, so we first :
\begin{align*}
	&dx=\frac{\partial x}{\partial r}dr+\frac{\partial x}{\partial \theta}d\theta+\frac{\partial x}{\partial \phi}d\phi\\
	&dy=\frac{\partial y}{\partial r}dr+\frac{\partial y}{\partial \theta}d\theta+\frac{\partial y}{\partial \phi}d\phi\\
	&dz=\frac{\partial z}{\partial r}dr+\frac{\partial z}{\partial \theta}d\theta+\frac{\partial z}{\partial \phi}d\phi
\end{align*}
Which gives :
\begin{align}
	&dx=\sin{\theta}\cos{\phi}dr+r\cos{\theta}\cos{\phi}d\theta-r\sin{\theta}\sin{\phi}d\phi \\
	&dy=\sin{\theta}\sin{\phi}dr+r\cos{\theta}\sin{\phi}d\theta+r\sin{\theta}\cos{\phi}d\phi \\
	&dz=\cos{\theta}dr-r\sin{\theta}d\theta
\end{align}
Back into (\ref{line-elem-schwarz}), looking inside the squared parenthesis, and using also (\ref{coord-transform-1})-(\ref{coord-transform-3}), we can replace as follow :
\begin{align*}
	-dt+\frac{x}{r}dx+&\frac{y}{r}dy+\frac{z}{r}dz =-dt\\
	&+\sin{\theta} \cos{\phi} (\sin{\theta} \cos{\phi} dr+r\cos{\theta} \cos{\phi} d\theta -r\sin{\theta} \sin{\phi} d\phi )\\
	&+\sin{\theta} \sin{\phi} (\sin{\theta} \sin{\phi} dr+r\cos{\theta} \sin{\phi} d\theta +r\sin{\theta} \cos{\phi} d\phi )\\
	&+\cos{\theta} (\cos{\theta} dr-\sin{\theta} d\theta )
\end{align*}
This equation is quite long but is is simplifiable. In order to see it  we will look at $dr$, $d\theta$, $d\phi$ separatly :
\begin{align*}
	dr :& \\
	&\sin^2{\theta}\cos^2{\phi}+\sin{\theta}^2\sin^2{\phi}+\cos^2{\phi}=\sin^2{\theta}+\cos^2{\theta}=1\\
	d\theta :& \\
	&r(\sin{\theta}\cos{\theta}\cos^2{\phi}+\sin{\theta}\cos{\theta}\sin^2{\phi}-\cos{\theta}\sin{\phi})=r(\sin{\theta}\cos{\theta}-\sin{\theta}\cos{\theta})=0\\
	d\phi :& \\
	&r(-\cos{\phi}\sin{\phi}\sin^2{\theta}+\sin^2{\theta}\cos{\phi}\sin{\phi})=0
\end{align*}
Hence if we replace in (\ref{line-elem-schwarz}) we get :
\begin{equation*}
	ds^2=-dt^2+dr^2+r^2d\Omega^2+\frac{2C}{r}(-dt+dr)^2
\end{equation*}
Expanding :
\begin{align*}
	&ds^2=-dt^2+dr^2+r^2d\Omega^2+\frac{2C}{r}(dt^2+dr^2-2dtdr)\\
	&ds^2=(\frac{2C}{r}-1)dt^2+(\frac{2C}{r}+1)dr^2+r^2d\Omega^2-\frac{4C}{r}dtdr
\end{align*}

\section{Deriving Kerr differently}
Now we want to use the same method we used for the Schwarzschild solution to find the Kerr solution.
The Kerr metric is another exact solution of the Einstein field equations found in 1963 by Roy P. Kerr.
It is a generalisation of the Schwarzschild solution, it describes an axially-symmetric rotating object.
This metric is often used to describe rotating black-holes.\\
/Image/
\subsection{Classical way of finding the Kerr solution}
Classical way of finding the Kerr metric is a complicated task.
There don't exist many ressources on this subject, and the few ressources that are available are clearly not accessible to undergraduate student.
They use complicated mathematical notion, and even a good mathematician would require some time to understand the proof.
Hence it would be interesting to find a simpler way to derive the Kerr solution.
\subsection{Elements of the Kerr-Shild for the Kerr case}
Once again we take a metric of the form :
\begin{equation*}
	g_{ij}=\eta_{ij}+2H(r,z)k_ik_j
\end{equation*}
This time we take $H$ to be dependant of $r$ and $z$ to take into account the axial symmetry.
We consider $r$ such that :
\begin{equation*}
	\frac{x^2+y^2}{r^2+a^2}+\frac{z^2}{r^2}=1
\end{equation*}
With $a$ the source angular momentum per unit mass.
We notice that as $aarrow 0$ we get the spherical $r$ as before.
And let $k$ be :
\begin{equation*}
	k_i=(-1,\frac{r x + a y}{r^2 + a^2},\frac{ry-ax}{r^2+a^2},\frac{z}{r})
\end{equation*}
Similarly we took $k_i$ to be a generalisation of the Schwarzschild $k_i$ as when $aarrow 0$ we get the Schwarzschild $k_i$.
\subsection{Einstein vacuum field equation}
This time finding the Einstein tensor is a bit more challenging.
The Schwarzschild case computed by hand is already a long calculation and adding a dregree a freedom in the Kerr case complicate even more the calculation.
Using a computer program like Mathematica, we can find the Einstein tensor components.
The simplest element of the Einstein tensor after simplifying as much as possible is $G_{00}$ :
\begin{align*}
	G_{00} = &\frac{1}{a^2 r^2z^2+r^6}(-2 a^2 r^3 z H^{(1,1)}(r,z)-a^2 r^2 z^2 H^{(0,2)}(r,z)+a^2 r^2(z^2-r^2) H^{(2,0)}(r,z)\\
	&-2 H(r,z) (z^2 (a^2z^2+r^4) H^{(0,2)}(r,z)+r ((a^2 z^2+r^4)(2 z H^{(1,1)}(r,z)+r H^{(2,0)}(r,z))\\
	&+4r^4H^{(1,0)}(r,z)-r^3)+4 r^4 z H^{(0,1)}(r,z))+a^2 z^4H^{(0,2)}(r,z)+2 a^2 r z^3 H^{(1,1)}(r,z)\\
	&+r^6(-H^{(0,2)}(r,z))+2 r^5 H^{(1,0)}(r,z)+r^4 z^2H^{(0,2)}(r,z)+4 r^4 z H^{(0,1)}(r,z)-4 r^4 H(r,z)^2)
\end{align*}
And even by carefully choosing the best looking term of the Einstein tensor it seems like we won't be able to solve the vacuum field equation $G_{ij}=0$.
We can do a little bit a better by looking at the Ricci tensor $R_{00}$ :
\begin{align*}
	R_{00}=-\frac{(a^2 z^2+r^4) H^{(0,2)}(r,z)+r
	((a^2+r^2) (2 z H^{(1,1)}(r,z)+r
	H^{(2,0)}(r,z))+2 r^2 H^{(1,0)}(r,z))}{a^2 z^2+r^4}
\end{align*}
It is a little bit better but still it is a second order PDE that can be quite difficult to solve.
As we start to see, solving the Einstein field equation for $H(r,z)$ with this metric, is not an easy task that you can exptect to complete in a few days.
It require a bit of research an work.
The good news is : Once you found a method to simplify the equation and make it solvable, the proof that you can find the Kerr metric using an Ansaltz in the Kerr-Schild is short and easy.
A good approach, the one that we are going to take, is to equate to zero a combination of parts of the Ricci tensor.
Indeed if we choose the following combination equal to $0$ :
\begin{equation*}
	(r^4+a^2z^2)^3(zR_{00}+rR_{03}+\frac{a^2+r^2}{rx+ay}(zR_{10}+rR_{13}))=0
\end{equation*}
We obtain the following 1st order PDE :
\begin{equation}
	(a^2 z^2+r^4) (z H^{(0,1)}(r,z)+rH^{(1,0)}(r,z))+(r^4-a^2 z^2) H(r,z)=0
\end{equation}
We hence finally obtain an equation that is nice looking, and that we can solve.

\section{Conclusion}

\end{document}