\documentclass[a4paper,12pt]{article}
\usepackage{amsmath}

\title{Kerr and Schwarzschild spacetimes derived differently}
\author{Adrien Sasportes}

\begin{document}

\maketitle
\pagenumbering{arabic}

\section{Introduction}

\section{Geometry and space-time}

\section{Vaidya type metric in GR}

\section{Deriving Schwarzschild differently}
In this section we will be interested in trying to find Schwarzschild differently using the general Kerr-Shild form :
\begin{equation*}
	g_{ij}=\eta_{ij}+2H(r)k_ik_j
\end{equation*}
The Schwarzschild solution is a really important solutionin general relativity. It was discovered a little more than a month after Einstein had formulated General Relativity and it describes the gravitationnal field outside a static spherical mass.
This solution can be used as an approximation for slowly rotating spherical astronomical object, like the Earth.
\subsection{Classical way of finding the Schwarzschild solution}
The usual way of finding the Schwarzschild metric usually start by writing the most general symmetric metric in vacuum :
\begin{equation*}
	ds^2=-e^{A(r)}dt^2+e^{B(r)}dr^2+r^2d\Omega^2
\end{equation*}
Then we compute the Einstein tensor, and we solve the Einstein vacuum field equation $G_{ab}=0$ by taking $G_{00}=0$ and $G_{11}=0$. Then we obtain the following equations :
\begin{align*}
	e^B+rB'-1&=0\\
	-e^B+rA'+1&=0
\end{align*}
From which we get :
\begin{equation*}
	e^A=e^{-B}=1-\frac{C}{r}
\end{equation*}
And thus we have the Schwarzschild solution :
\begin{equation*}
	ds^2=-(1-\frac{C}{r})dt^2+(1-\frac{C}{r})^{-1}dr^2+r^2d\Omega^2
\end{equation*}
This method of finding the Schwarzschild solution is work perfectly way but still it is always interesting to find an alternative way to find it.
\subsection{Elements of the Kerr-Shild for the Schwarzschild case}
We now want to focus on a metric of the following form $g_{ij}=\eta_{ij}+2H(r)k_ik_j$.
First in the Schwarzschild case we consider a spherical object so we consider some $r$ such that $r^2=x^2+y^2+z^2$.
$\eta_{ij}$ is known as the Minkowski metric defines as $\eta_{ij}=diag(+1,+1,+1,-1)$
As for the $H$ we guess that it would only depend on $r$.
We define $k_i=(-1,\frac{x}{r},\frac{y}{r},\frac{z}{r})$, a null geodesic.
We notice that $k_ik^i=0 <=> r^2=x^2+y^2+z^2$.
Looking at the $x$, $y$ and $z$ components of $k$, $k$ looks like a radial vector, but we have a $-1$ in the $t$ direction.
\subsection{Einstein vacuum field equation}
From the formula we derived for $G_{ij}$ for metric of this form we can fairly easily obtain the Einstein tensor.
Looking at $G_{i0}, i=1,2,3$ we get :
\begin{equation}
	G_{i0}=-\frac{8 H(r) \left(r^2 H'(r)+rH(r)\right)}{\left(r^2\right)^{3/2}}
\end{equation}
Our goal is to find $H(r)$ that solves the Einstein vacuum field equation $G_{ij}=0$ which in our case is $G_{i0}=0, i=1,2,3$ or :
\begin{equation}
	-\frac{8 H(r) \left(r^2 H'(r)+rH(r)\right)}{\left(r^2\right)^{3/2}}=0
\end{equation}
Which in general is equivalent to :
\begin{equation}
	rH'(r)+H(r)=0
\end{equation}
This equation is quite nice and can easily be solved as follow :
\begin{align*}
	rH'(r)+H(r)&=0\\
	r\frac{d}{dr}H(r)&=-H(r)\\
	\frac{1}{H(r)}\frac{d}{dr}H(r)&=-\frac{1}{r}\\
	\frac{1}{H(r)}dH(r)&=-\frac{1}{r}dr \;\\
	&\text{, integrating,}\\
	log(H(r))&=-\log{r}+C \;\\
	&\text{, with C constant}\\
	H(r)&=\frac{C}{r}
\end{align*}
Giving us :
\begin{equation}
	g_{ij}=\eta_{ij}+\frac{2C}{r}k_ik_j
\end{equation}
This is the Schwarzschild solution. As it is not so obvious in this form, in the next sub-section, we will put in back in the "classical form".
\subsection{Back to the classical form of the Schwarzschild solution}
In order to be able to find the classical Schwarzschild form we should first write the line element of the metric we found explcitly :
\begin{equation*}
	ds^2=g_{ij}dX^idX^j
\end{equation*}
\begin{equation}
	ds^2=-dt^2+dx^2+dy^2+dz^2+\frac{2C}{r}(-dt+\frac{x}{r}dx+\frac{y}{r}dy+\frac{z}{r}dz)^2
\end{equation}
In order to find the classical form of the Schwarzschild metric, we will right this line element in spherical coordinates $X=\{t,r,\theta,\phi\}$ define as :
\begin{align*}
	&t=t\\
	&r^2=x^2+y^2+z^2
\end{align*}
\begin{align}
	&x=r\sin{\theta}\cos{\phi}\label{coord-transform-1}\\
	&y=r\sin{\theta}\sin{\phi}\\
	&z=r\cos{\theta}\label{coord-transform-3}
\end{align}
As we have constructed $g_{ij}$ to be the sum of the Minkowski metric and another element we can immediatly replace the Minkowski part of our metric with the Minkowski metric in spherical coordinates :
\begin{equation} \label{line-elem-schwarz}
	ds^2=-dt^2+dr^2+r^2d\Omega^2\;+\frac{2C}{r}(-dt+\frac{x}{r}dx+\frac{y}{r}dy+\frac{z}{r}dz)^2
\end{equation} 
Now let's focus on $\frac{2C}{r}(-dt+\frac{x}{r}dx+\frac{y}{r}dy+\frac{z}{r}dz)^2$.\\
First we want to write $dx$, $dy$, $dz$ as a linear sum of $dr$, $d\theta$, $d\phi$, so we first right :
\begin{align*}
	&dx=\frac{\partial x}{\partial r}dr+\frac{\partial x}{\partial \theta}d\theta+\frac{\partial x}{\partial \phi}d\phi\\
	&dy=\frac{\partial y}{\partial r}dr+\frac{\partial y}{\partial \theta}d\theta+\frac{\partial y}{\partial \phi}d\phi\\
	&dz=\frac{\partial z}{\partial r}dr+\frac{\partial z}{\partial \theta}d\theta+\frac{\partial z}{\partial \phi}d\phi
\end{align*}
Which gives :
\begin{align}
	&dx=\sin{\theta}\cos{\phi}dr+r\cos{\theta}\cos{\phi}d\theta-r\sin{\theta}\sin{\phi}d\phi \\
	&dy=\sin{\theta}\sin{\phi}dr+r\cos{\theta}\sin{\phi}d\theta+r\sin{\theta}\cos{\phi}d\phi \\
	&dz=\cos{\theta}dr-r\sin{\theta}d\theta
\end{align}
Back into (\ref{line-elem-schwarz}), looking inside the squared parenthesis, and using also (\ref{coord-transform-1})-(\ref{coord-transform-3}), we can replace as follow :
\begin{align*}
	-dt+\frac{x}{r}dx+&\frac{y}{r}dy+\frac{z}{r}dz =-dt\\
	&+\sin{\theta} \cos{\phi} (\sin{\theta} \cos{\phi} dr+r\cos{\theta} \cos{\phi} d\theta -r\sin{\theta} \sin{\phi} d\phi )\\
	&+\sin{\theta} \sin{\phi} (\sin{\theta} \sin{\phi} dr+r\cos{\theta} \sin{\phi} d\theta +r\sin{\theta} \cos{\phi} d\phi )\\
	&+\cos{\theta} (\cos{\theta} dr-\sin{\theta} d\theta )
\end{align*}
This equation is quite long but is is simplifiable. In order to see it  we will look at $dr$, $d\theta$, $d\phi$ separatly :
\begin{align*}
	dr :& \\
	&\sin^2{\theta}\cos^2{\phi}+\sin{\theta}^2\sin^2{\phi}+\cos^2{\phi}=\sin^2{\theta}+\cos^2{\theta}=1\\
	d\theta :& \\
	&r(\sin{\theta}\cos{\theta}\cos^2{\phi}+\sin{\theta}\cos{\theta}\sin^2{\phi}-\cos{\theta}\sin{\phi})=r(\sin{\theta}\cos{\theta}-\sin{\theta}\cos{\theta})=0\\
	d\phi :& \\
	&r(-\cos{\phi}\sin{\phi}\sin^2{\theta}+\sin^2{\theta}\cos{\phi}\sin{\phi})=0
\end{align*}
Hence if we replace in (\ref{line-elem-schwarz}) we get :
\begin{equation*}
	ds^2=-dt^2+dr^2+r^2d\Omega^2+\frac{2C}{r}(-dt+dr)^2
\end{equation*}
Expanding :
\begin{align*}
	&ds^2=-dt^2+dr^2+r^2d\Omega^2+\frac{2C}{r}(dt^2+dr^2-2dtdr)\\
	&ds^2=(\frac{2C}{r}-1)dt^2+(\frac{2C}{r}+1)dr^2+r^2d\Omega^2-\frac{4C}{r}dtdr
\end{align*}

\section{Deriving Kerr differently}
Now we want to use the same method we used for the Schwarzschild solution to find the Kerr solution.
The Kerr metric is another exact solution of the Einstein field equations found in 1963 by Roy P. Kerr.
It is a generalisation of the Schwarzschild solution, it describes an axially-symmetric rotating object.
This metric is often used to describe rotating black-holes.\\
/Image/
\subsection{Classical way of finding the Kerr solution}

\section{Conclusion}

\end{document}