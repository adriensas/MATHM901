\documentclass[a4paper,12pt]{article}
\usepackage{amsmath}

\title{Kerr and Schwarzschild spacetimes derived differently}
\author{Adrien Sasportes}

\begin{document}

\maketitle
\pagenumbering{arabic}

\section{Introduction}

\section{Geometry and space-time}

\section{Vaidya type metric in GR}

\section{Deriving Schwarzschild differently}
In this section we will be interested in trying to find Schwarzschild differently using the general Kerr-Shild form :
\begin{equation*}
	g_{ij}=\eta_{ij}+2H(r)k_ik_j
\end{equation*}
The Schwarzschild solution is a really important solutionin general relativity. It was discovered a little more than a month after Einstein had formulated General Relativity and it describes the gravitationnal field outside a static spherical mass.
This solution can be used as an approximation for slowly rotating spherical astronomical object, like the Earth.
\subsection{Classical way of finding the Schwarzschild solution}
The usual way of finding the Schwarzschild metric usually start by writing the most general symmetric metric in vacuum :
\begin{equation*}
	ds^2=-e^{A(r)}dt^2+e^{B(r)}dr^2+r^2d\Omega^2
\end{equation*}
Then we compute the Einstein tensor, and we solve the Einstein vacuum field equation $G_{ab}=0$ by taking $G_{00}=0$ and $G_{11}=0$. Then we obtain the following equations :
\begin{align*}
	e^B+rB'-1&=0\\
	-e^B+rA'+1&=0
\end{align*}
From which we get :
\begin{equation*}
	e^A=e^{-B}=1-\frac{C}{r}
\end{equation*}
And thus we have the Schwarzschild solution :
\begin{equation*}
	ds^2=-(1-\frac{C}{r})dt^2+(1-\frac{C}{r})^{-1}dr^2+r^2d\Omega^2
\end{equation*}
This method of finding the Schwarzschild solution is work perfectly way but still it is always interesting to find an alternative way to find it.
\subsection{Elements of the Kerr-Shild for the Schwarzschild case}
We now want to focus on a metric of the following form $g_{ij}=\eta_{ij}+2H(r)k_ik_j$.
First in the Schwarzschild case we consider a spherical object so we consider some $r$ such that $r^2=x^2+y^2+z^2$.
$\eta_{ij}$ is known as the Minkowski metric defines as $\eta_{ij}=diag(+1,+1,+1,-1)$
As for the $H$ we guess that it would only depend on $r$.
We define $k_i=(-1,\frac{x}{r},\frac{y}{r},\frac{z}{r})$, a null geodesic.
We notice that $k_ik^i=0 <=> r^2=x^2+y^2+z^2$.
Looking at the $x$, $y$ and $z$ components of $k$, $k$ looks like a radial vector, but we have a $-1$ in the $t$ direction.
\subsection{Einstein vacuum field equation}
From the formula we derived for $G_{ij}$ for metric of this form we can fairly easily obtain the Einstein tensor.
Looking at $G_{i0}, i=1,2,3$ we get :
\begin{equation}
	G_{i0}=-\frac{8 H(r) \left(r^2 H'(r)+rH(r)\right)}{\left(r^2\right)^{3/2}}
\end{equation}
Our goal is to find $H(r)$ that solves the Einstein vacuum field equation $G_{ij}=0$ which in our case is $G_{i0}=0, i=1,2,3$ or :
\begin{equation}
	-\frac{8 H(r) \left(r^2 H'(r)+rH(r)\right)}{\left(r^2\right)^{3/2}}=0
\end{equation}
Which in general is equivalent to :
\begin{equation}
	rH'(r)+H(r)=0
\end{equation}
This equation is quite nice and can easily be solved as follow :
\begin{align*}
	&rH'(r)+H(r)=0\\
	&r\frac{d}{dr}H(r)=-H(r)\\
	&\frac{1}{H(r)}\frac{d}{dr}H(r)=-\frac{1}{r}\\
	&\frac{1}{H(r)}dH(r)=-\frac{1}{r}dr\\
	&log(H(r))=-log(r)+C\\
	&H(r)=\frac{C}{r}
\end{align*}
Giving us :
\begin{equation}
	g_{ij}=\eta_{ij}+\frac{2C}{r}k_ik_j
\end{equation}
This is the Schwarzschild solution. As it is not so obvious in this form, in the next sub-section, we will put in back in the "classical form".
\subsection{Back to the classical form of the Schwarzschild solution}

\section{Deriving Kerr differently}

\section{Conclusion}

\end{document}