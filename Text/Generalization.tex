\documentclass{article}
\usepackage{amsmath}

\begin{document}
\pagenumbering{arabic}
\section{Generalization}
    \subsection{Introduction}
        While we computed all the different tensors, one can notice some kind of pattern.
        Moreover the complexity increase substantially while we add some freedom.
        A good idea would be to introduce a general metric from which we hope to find some interesting general form for the Christoffel Symbol, the Ricci Tensor and the Einstein tensor.
    \subsection{The metric}
        Let us introduce the metric :
        \begin{equation}
            g_{ij}=\eta_{ij}+Hk_ik_j
        \end{equation}
        with $k_i$ any null geodesic and $H$ any function $H:=H(t,x,y,z)$.
        The first thing we will need to find is the inverse of this mmetric.
        Indeed we will need it to compute the Christoffel symbols.
        We guess from our previous calculus an inverse of the form :
        \begin{equation}
            g^{ij}=\eta^{ij}-Hk^ik^j
        \end{equation}
        We can quickly check it as follow :
        \begin{equation*}
            g^{ij}g_{ij}= ...
            ...
        \end{equation*}
    \subsection{The Christoffel symbols}
        As already said computing the Christoffel symbol can quickly be really painful.
        It would be nice if we could find some general easier form for the Christoffel symbol.
        Recalling $\eta^i_{kl}=\frac{1}{2}g^{im}(g_{mk,l}+g_{ml,k}-g{kl,m})$ and replacing with our format of metric.
\end{document}